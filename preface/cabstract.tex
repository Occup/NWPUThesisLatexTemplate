
%%%%%%%%%%%%%%%%%%%%%%%%%%%%%%%%%%%%%%%%%%%%%%%%%%%%%%%%%%%%%%%%%%%%%%%%%
\renewcommand{\baselinestretch}{1.5}
\fontsize{12pt}{13pt}\selectfont

\chapter*{摘~~~~要}

\vspace{1em}

太阳敏感器作为卫星姿态确定与控制系统中最为重要的敏感器已经得到十
分广泛的应用,也发展出了各种类型的太阳敏感器以满足不同的应用需求。
随着微纳卫星的设计不断创新与应用领域不断拓展,其设计制造的每一个
具体环节形成具体的操作标准这一需求愈发迫切。

本文总结了当前阶段适用于微小卫星的太阳敏感器的发展现状以及主要技术
指标,列举出当前阶段这一研究课题的具体产品,分析总结了前人提出的一
些标定策略与实验方法,结合本文对四象限模拟式太阳敏感器的分析完成了
标定实验与标定实验操作标准的制定。

通常而言普通模拟太阳敏感器的精度不高,抗干扰能力较弱,
本文即对原有的模拟式四象限太阳敏感器的设计思想与工作原理进行研究,
深入探讨了四象限太阳敏感器测量误差的来源并对其进行定量分析,推导出
了误差补偿公式,设计针对小体积、低功耗、大视场角、高精度等特点的太阳
敏感器的测量标定方法,使其具有了更高的测量精度,从而能够充分满足微小
卫星的各类应用要求。

\vspace{0.1in}
\noindent {\CJKfamily{zhhei} 关键词:四象限模拟式太阳敏感器$\quad$误差分析$\quad$误差标定$\quad$操作规程} 